\subsection{topic 用法}
主題是在節點之間以及系統不同部分之間移動數據的主要方式之一。

可以用-h來查詢topic可以有哪些操作如下。
\begin{enumerate}
\item  bw     :顯示主題使用的頻寬
\item  delay  :從標題中的時間戳顯示主題的延遲
\item  echo   :從主題輸出訊息
\item  find   :輸出給定類型的可用主題列表
\item  hz     :將平均發布平率列印到螢幕上
\item  info   :列印主題的訊息
\item  list   :輸出可用主題的列表
\item  pub    :向主題發布訊息
\item  type   :列印主題的類型
\end{enumerate}

\subsection{list}
在新終端機中執行該命令將傳回系統中目前活動的所有主題的清單:
 \begin{verbatim}
     ros2 topic list 
     ----------------
     /parameter_events
     /rosout
     /turtle1/cmd_vel
     /turtle1/color_sensor
     /turtle1/pose
\end{verbatim}

\subsection{echo}
若要查看某個主題上發布的數據,請使用:

 \begin{verbatim}
     ros2 topic echo <topic_name>
     ros2 topic echo /turtle1/cmd_vel
     ---------------------------------
     linear:
     x: 2.0
     y: 0.0
     z: 0.0
     angular:
     x: 0.0
     y: 0.0
     z: 0.0
     ---
\end{verbatim}
\subsection{info}
可以查詢主題的訂閱與發布資訊。
\begin{verbatim}
    ros2 topic info /turtle1/cmd_vel
    --------------------------------
    Type: geometry_msgs/msg/Twist
    Publisher count: 1
    Subscription count: 2
\end{verbatim}

\subsection{interface}
節點使用訊息透過主題發送資料。發布者和訂閱者必須發送和接收相同類型的訊息才能進行通訊。

我們可以透過info去查詢topic的tyep,在由interface show 來查詢格式。

\newpage
\begin{verbatim}
    ros2 interface show geometry_msgs/msg/Twist
    -------------------------------------------
    # This expresses velocity in free space broken 
    into its linear and angular parts.

    Vector3  linear
    float64 x
    float64 y
    float64 z
    Vector3  angular
    float64 x
    float64 y
    float64 z
\end{verbatim}

\subsection{pub}
現在您已經有了訊息結構,
用以下命令直接從命令列將資料發佈到主題:

'<args>'是您將傳遞到主題的實際數據,採用您在上一節中剛剛發現的結構。


\begin{verbatim}
    ros2 topic pub <topic_name> <msg_type> '<args>'
    ros2 topic pub --once /turtle1/cmd_vel \ 
        geometry_msgs/msg/Twist \ 
        "{linear: {x: 2.0, y: 0.0, z: 0.0},\ 
        angular: {x: 0.0, y: 0.0, z: 1.8}}"
\end{verbatim}

以上我們我們了解如何查尋主題的發布的內容,與如何對主題輸入資料。

