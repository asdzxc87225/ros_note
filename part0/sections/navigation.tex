這部份我還沒研究完,我看dblanding大大這部份只有寫參數設定,根
啟動檔,這邊應該除了dblanding的github還要在查看其他資料。
\subsection{導航參數文件}
\begin{table}[ht]
\centering
\begin{tabularx}{\textwidth}{|l|X|}
\hline
\textbf{檔案名稱}                     & \textbf{用途}                                                   \\ \hline
base\_local\_planner\_params.yaml  & 配置基礎局部規劃器(local planner)的參數                         \\ \hline
costmap\_common\_params.yaml       & 配置共用的地圖(costmap)參數,用於全局和局部 costmap             \\ \hline
dwa\_local\_planner\_params.yaml   & 配置動態窗口局部規劃器(Dynamic Window Approach)的參數         \\ \hline
global\_costmap\_params.yaml       & 配置全局 costmap 的參數                                         \\ \hline
global\_planner\_params.yaml       & 配置全局規劃器(global planner)的參數                           \\ \hline
local\_costmap\_params.yaml        & 配置局部 costmap 的參數                                         \\ \hline
move\_base\_params.yaml            & 配置 move\_base 模組的參數,用於整合全局和局部規劃器以及 costmap \\ \hline
navfn\_global\_planner\_params.yaml& 配置基於 Dijkstra 或 A* 算法的全局規劃器(navfn)的參數          \\ \hline
\end{tabularx}
\end{table}


