
\subsection{操作方法}

本次使的lidar是RPLIDAR-A1,再有ros2的套件輔助下我們可以教簡
單使用。
\begin{enumerate}
    \item 使用git把原始碼安裝在工作目錄
    \item 編譯原始碼
    \item 執行測試
\end{enumerate}
\subsubsection{安裝編譯原始碼}
動作流程如下。
\begin{tcolorbox}
\begin{verbatim}
    cd ~/ros2_ws/src
    git clone https://github.com/Slamtec/sllidar_ros2
    cd ../ 
    rosdep install -i --from-path src --rosdistro humble -y
    colcon build --packages-select sllidar_ros2
\end{verbatim}
\end{tcolorbox}
\begin{enumerate}
    \item 移動到工作目錄的原始碼資料夾
    \item 下載lidar的套件
    \item 回到工作根目錄
    \item 檢查相依程式
    \item 開始編譯
\end{enumerate}
\subsubsection{測試與使用}
編譯完原始碼的話基本上就可以使用了,使用方法如下
\begin{tcolorbox}
\begin{verbatim}
    source install/setup.zsh    
    ros2 launch sllidar_ros2 sllidar_a1_launch.py
    #可以選折下面有視覺化設定的
    ros2 launch sllidar_ros2 view_sllidar_a1_launch.py
\end{verbatim}
\end{tcolorbox}
\begin{enumerate}
    \item 啟用ros2環境
    \item 開啟lidar的請動檔
\end{enumerate}
這邊要注意使用的光達型號不同型號有對應的啟動檔,也可以直接
使用可是化的啟動檔view開頭的那個。
\subsection{檢視資料}
接下來就以觀察這個套件有哪些要注意的地方。

\begin{itemize}
    \item 開啟的節點
    \item 主題的發布與訂閱
    \item 資料的格式
    \item 節點參數
\end{itemize}
\subsubsection{節點說明}
這個光達啟動檔產生的節點為/sllidar\_node,我們可以用info查
詢相關訊息。
\begin{tcolorbox}
\begin{verbatim}
    ros2 node info /sllidar\_node
\end{verbatim}
\end{tcolorbox}
這樣可以看到下面有關node的通訊狀況,包含他會訂閱/發布哪些主
題與服務根動作。
\begin{itemize}
    \item 訂閱器(Subscribers):
        \begin{itemize}
            \item /parameter\_events:訂閱參數事件的消息,
                類型為 rcl\_interfaces/msg/ParameterEvent。
        \end{itemize}
    \item 發布器(Publishers):

        \begin{itemize}
            \item /parameter\_events:發布參數事件的消息,
                類型為 rcl\_interfaces/msg/ParameterEvent。
            \item /rosout:發布ROS的日誌消息,類型為 
                rcl\_interfaces/msg/Log。
            \item /scan:發布激光雷達掃描消息,類型為 
                sensor\_msgs/msg/LaserScan。
        \end{itemize}
    \item 服務伺服器(Service Servers):
        \begin{itemize}
            \item /sllidar\_node/describe\_parameters:提供描
                述參數的服務,類型為 \\
                rcl\_interfaces/srv/DescribeParameters。
            \item /sllidar\_node/get\_parameter\_types:提供獲
                取參數類型的服務,類型為 \\
                rcl\_interfaces/srv/GetParameterTypes。
            \item /sllidar\_node/get\_parameters:提供獲取參數
                值的服務,類型為 \\
                rcl\_interfaces/srv/GetParameters。
            \item /sllidar\_node/list\_parameters:
                提供列出參數的服務,類型為\\
                rcl\_interfaces/srv/ListParameters。
            \item /sllidar\_node/set\_parameters:
                提供設置參數值的服務,類型為 \\
                rcl\_interfaces/srv/SetParameters。
            \item /sllidar\_node/set\_parameters\_atomically:
                提供原子設置多個參數值的服務,類型為 \\
                rcl\_interfaces/srv/SetParametersAtomically。
            \item /start\_motor:提供啟動馬達的服務,
                類型為 std\_srvs/srv/Empty。
            \item /stop\_motor:提供停止馬達的服務,類型為 
                std\_srvs/srv/Empty。
        \end{itemize}
    \item 服務客戶端(Service Clients):(未提供)
    \item 動作伺服器(Action Servers):(未提供)
    \item 動作客戶端(Action Clients):(未提供)
\end{itemize}

這些功能使得sllidar\_node節點能夠與其他節點進行通訊和交互操作,從而實現控制和獲取激光雷達的功能。

\subsubsection{topic查詢}
首先要確認/scan的訊息。
\begin{tcolorbox}
\begin{verbatim}
    ros2 lidar info /scan
------------------------------------------
Type: sensor_msgs/msg/LaserScan
Publisher count: 1
Subscription count: 0
\end{verbatim}
\end{tcolorbox}
我們看得出來他的,訂閱與發不的狀況根他的數據界面。
再來我們可以用echo來監聽這個主題。
\begin{tcolorbox}
\begin{verbatim}
    ros2 topic echo /scan --once
-----------------------------------
    header:
  stamp:
    sec: 1710051588
    nanosec: 187130763
  frame_id: laser
angle_min: -3.1241390705108643
angle_max: 3.1415927410125732
angle_increment: 0.005806980188935995
time_increment: 0.0001362734183203429
scan_time: 0.1470390260219574
range_min: 0.15000000596046448
range_max: 12.0
ranges:
.......
intensities:
\end{verbatim}
\end{tcolorbox}
有關各個數值的意義如下
\begin{itemize}
    \item header:包含時間戳記(stamp)和框架 ID(frame\_id),用於確定訊息的時間和座標系。
    \item stamp:時間戳記,以秒(sec)和納秒(nanosec)表示。
    \item frame\_id:訊息的座標系標識符,在這裡是 laser。
    \item angle\_min:激光雷達掃描的最小角度(弧度)。
    \item angle\_max:激光雷達掃描的最大角度(弧度)。
    \item angle\_increment:每個激光束之間的角度增量(弧度)。
    \item time\_increment:每個激光束的時間增量(秒)。
    \item scan\_time:完成一次完整掃描所需的時間(秒)。
    \item range\_min:激光雷達能夠測量的最小距離(公尺)。
    \item range\_max:激光雷達能夠測量的最大距離(公尺)。
    \item ranges:每個激光束測量的距離值陣列。
    \item intensities:每個激光束的強度值陣列(有些激光雷達可以測量強度,但並非所有)。
\end{itemize}
\subsubsection{參數查詢}
接下來就是參數的訊息了,透過param dump來查看

\begin{tcolorbox}
\begin{verbatim}
    ros2 param dump /sllidar_node
-------------------------------------------
/sllidar_node:
  ros__parameters:
    angle_compensate: true
    channel_type: serial
    frame_id: laser
    inverted: false
    qos_overrides:
      /parameter_events:
        publisher:
          depth: 1000
          durability: volatile
          history: keep_last
          reliability: reliable
    scan_frequency: 10.0
    scan_mode: ''
    serial_baudrate: 115200
    serial_port: /dev/ttyUSB0
    tcp_ip: 192.168.0.7
    tcp_port: 20108
    udp_ip: 192.168.11.2
    udp_port: 8089
    use_sim_time: false
\end{verbatim}
\end{tcolorbox}
這是sllidar\_node節點的參數設定。讓我們逐個解釋:
\begin{enumerate}
    \item angle\_compensate:角度補償,設置為true表示啟用角度補償。
    \item channel\_type:通道類型,這裡設置為serial。
    \item frame\_id:訊息的座標系標識符,在這裡是laser。
    \item inverted:是否反向,設置為false表示不反向。
    \item qos\_overrides:QoS覆蓋設定,這裡設置了一些QoS參數。
    \item scan\_frequency:掃描頻率,設置為10.0 Hz。
    \item serial\_baudrate:串口波特率,設置為115200。
    \item serial\_port:串口端口,設置為/dev/ttyUSB0。
    \item tcp\_ip:TCP IP地址,設置為192.168.0.7。
    \item tcp\_port:TCP端口,設置為20108。
    \item udp\_ip:UDP IP地址,設置為192.168.11.2。
    \item udp\_port:UDP端口,設置為8089。
    \item use\_sim\_time:使用模擬時間,設置為false表示不使用模擬時間。
\end{enumerate}

這些參數描述了sllidar\_node節點的配置,包括通訊方式、座標
系、掃描頻率等,我們是走serial通訊的,所以要
注意port的參數未來接上的設定。

\subsection{資料視覺化}
這裡我們用rviz可以更清楚的看到光達所掃掉的資料。

\begin{enumerate}
    \item 打開終端機
    \item 輸入rviz2
    \item 設定fixed frame = laser
    \item 在add加入laser scan的資料
\end{enumerate}
