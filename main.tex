% !TeX program = xelatex
\documentclass{article}

\usepackage{xeCJK}
\usepackage{listings}
\usepackage{graphicx}
\usepackage{geometry}
\usepackage{hyperref}

\setCJKmainfont{Noto Sans Mono CJK TC} % 設定中文字體
\setCJKmonofont{Noto Sans Mono CJK TC} % 設定等寬字體
\geometry{a4paper, left=2cm, right=2cm, top=2cm, bottom=2cm}
\hypersetup{
    colorlinks=true,    % 将书签文字用颜色表示
    linkcolor=blue,     % 内部链接的颜色
    citecolor=green,    % 引用的颜色
    filecolor=magenta,  % 文件链接的颜色
    urlcolor=cyan       % URL的颜色
}


\title{ROS 筆記}
\author{luu}
\date{} % 添加日期

\begin{document}

\maketitle
\tableofcontents

\part{CLI 操作}
\section{CLI 設定}
% cli_settings.tex
在運行ROS 2的指令時需要開啟環境,所以要先使用source來啟用。
\begin{verbatim}
source /opt/ros/humble/setup.zsh
\end{verbatim}
但是每次都要執行會有點多餘,所以可以寫在shell的設定檔內。
\begin{verbatim}
echo "source /opt/ros/humble/setup.zsh" >> ~/.zshrc
\end{verbatim}


\section{練習用的模擬器}
% simulator_practice.tex
Turtlesim是一個用於學習ROS 2的輕量級模擬器。它說明了ROS 2在最基本層面上的功能,讓您了解稍後將使用真實機器人或機器人模擬做什麼。

ROS 2工具是使用者管理、反思以及與ROS系統互動的方式。它支援針對系統及其操作的不同方面的多個命令。人們可以使用它來啟動節點、設定參數、監聽主題等等。ROS 2工具是核心ROS 2安裝的一部分。

\subsection*{安裝}
安裝東西的時候記得要更新來源。
\begin{verbatim}
sudo apt update
sudo apt install ros-humble-turtlesim
sudo apt install ~nros-humble-rqt"*
\end{verbatim}
檢查軟體包是否已安裝:
\begin{verbatim}
ros2 pkg executables turtlesim
#--------------------
turtlesim draw_square
turtlesim mimic
turtlesim turtle_teleop_key
turtlesim turtlesim_node
\end{verbatim}

\subsection*{運行}
烏龜模擬器
\begin{verbatim}
ros2 run turtlesim turtlesim_node
ros2 pkg executables turtlesim
\end{verbatim}

rqt是ROS 2的圖形使用者介面(GUI)工具。在rqt中完成的所有操
作都可以在命令列上完成,但rqt提供了一種更用戶友好的方式來操作ROS
2元素。
\begin{verbatim}
rqt
\end{verbatim}


\section{節點}
ROS 中的每個節點都應負責單一的模組化目的,例如控制車輪馬達或發布來自雷射測距儀的感測器資料。每個節點都可以透過主題、服務、操作或參數從其他節點發送和接收資料。

完整的機器人系統由許多協同工作的節點組成。在 ROS 2 中,單一可執行檔(C++ 程式、Python 程式等)可以包含一個或多個節點。

\subsection{ros2 直行方法}
run 的使用方式如下
\begin{verbatim}
    ros2 run <package_name> <executable_name>
    -------------------------------------------
    usage: ros2 run [-h] [--prefix PREFIX]
    package_name executable_name ...

    Run a package specific executable

    positional arguments:
    package_name     Name of the ROS package
    executable_name  Name of the executable
    argv             Pass arbitrary arguments to the executable

    options:
    -h, --help       show this help message and exit
    --prefix PREFIX  Prefix command, which should go before the
    executable. Command must be wrapped in quotes
    if it contains spaces (e.g. --prefix 'gdb -ex
    run --args').
\end{verbatim}
操作流程如下
\begin{enumerate}
    \item 開啟第一個terminal
    \item 開啟turtlesim turtlesim\_node
    \item 開啟第二個terminal
    \item 檢查節點裝況
\end{enumerate}
開啟模擬視窗
\begin{verbatim}
    ros2 run <package_name> <executable_name>
    ros2 run turtlesim turtlesim_node
\end{verbatim}

檢查節點
\begin{verbatim}
    ros2 node list
\end{verbatim}
開啟控制器
\begin{verbatim}
    ros2 run turtlesim turtle_teleop_key
\end{verbatim}
\subsection{重新映射}
重新映射可讓您將預設節點屬性(例如節點名稱、主題名稱、服務名稱等)重新指派給自訂值。
\begin{verbatim}
ros2 run turtlesim turtlesim_node --ros-args --remap \
__node:=my_turtle
\end{verbatim}
這時後會看到我們映射的節點
\begin{verbatim}
    /my_turtle
    /turtlesim
    /teleop_turtle
\end{verbatim}

\subsection{節點訊息}
    可以用info 來查詢節點的資料,方法如下。
\begin{verbatim}
ros2 node info <node_name>
ros2 node info /my_turtle
\end{verbatim}
可以看到如下的連線資料
\begin{enumerate}
    \item  Subscribers:訂閱者
    \item  Publishers:發布者
    \item  Service Servers:服務伺服器
    \item  Service Clients:服務客戶端
    \item  Action Servers:動作伺服器
    \item  Action Clients:動作客戶端
\end{enumerate}
\subsubsection{訂閱者(Subscribers):}

訂閱者是ROS中的一個元件,它可以接收特定主題(Topic)的訊息。
主題是ROS中一種訊息傳遞機制,允許節點(ROS中的程式)透過發佈和訂閱訊息進行通訊。
\subsubsection{發佈者(Publishers):}

發佈者是ROS中的一個元件,它能夠向特定主題發佈訊息。
發佈者節點生成訊息並將其發送到相應的主題,使其他訂閱者節點能夠接收這些訊息。
\subsubsection{服務伺服器(Service Servers):}

服務伺服器提供了一種不同的通訊機制,稱為服務(Service),允許節點提供某種特定的功能或服務。
當一個節點請求服務時,服務伺服器執行相應的任務,並返回結果。
\subsubsection{服務客戶端(Service Clients):}

服務客戶端是一個節點,它能夠向服務伺服器發送請求,並等待伺服器返回結果。
這種通訊模式通常用於需要特定服務或功能的節點之間的交互。
\subsubsection{動作伺服器(Action Servers):}

動作伺服器是ROS中處理長時間執行任務的一種機制。它允許異步執行,並提供反饋。
與服務不同,動作可以週期性地提供反饋,而不僅僅是單一的請求和回應。
\subsubsection{動作客戶端(Action Clients):}

動作客戶端是一個節點,它可以向動作伺服器發送請求,並接收反饋和結果。
這通常用於需要處理較長時間執行的任務,而服務模型則適用於簡短的請求和回應。

\section{主題}
\subsection{topic 用法}
主題是在節點之間以及系統不同部分之間移動數據的主要方式之一。

可以用-h來查詢topic可以有哪些操作如下。
\begin{enumerate}
\item  bw     :顯示主題使用的頻寬
\item  delay  :從標題中的時間戳顯示主題的延遲
\item  echo   :從主題輸出訊息
\item  find   :輸出給定類型的可用主題列表
\item  hz     :將平均發布平率列印到螢幕上
\item  info   :列印主題的訊息
\item  list   :輸出可用主題的列表
\item  pub    :向主題發布訊息
\item  type   :列印主題的類型
\end{enumerate}

\subsection{list}
在新終端機中執行該命令將傳回系統中目前活動的所有主題的清單:
 \begin{verbatim}
     ros2 topic list 
     ----------------
     /parameter_events
     /rosout
     /turtle1/cmd_vel
     /turtle1/color_sensor
     /turtle1/pose
\end{verbatim}

\subsection{echo}
若要查看某個主題上發布的數據,請使用:

 \begin{verbatim}
     ros2 topic echo <topic_name>
     ros2 topic echo /turtle1/cmd_vel
     ---------------------------------
     linear:
     x: 2.0
     y: 0.0
     z: 0.0
     angular:
     x: 0.0
     y: 0.0
     z: 0.0
     ---
\end{verbatim}
\subsection{info}
可以查詢主題的訂閱與發布資訊。
\begin{verbatim}
    ros2 topic info /turtle1/cmd_vel
    --------------------------------
    Type: geometry_msgs/msg/Twist
    Publisher count: 1
    Subscription count: 2
\end{verbatim}

\subsection{interface}
節點使用訊息透過主題發送資料。發布者和訂閱者必須發送和接收相同類型的訊息才能進行通訊。

我們可以透過info去查詢topic的tyep,在由interface show 來查詢格式。

\newpage
\begin{verbatim}
    ros2 interface show geometry_msgs/msg/Twist
    -------------------------------------------
    # This expresses velocity in free space broken 
    into its linear and angular parts.

    Vector3  linear
    float64 x
    float64 y
    float64 z
    Vector3  angular
    float64 x
    float64 y
    float64 z
\end{verbatim}

\subsection{pub}
現在您已經有了訊息結構,
用以下命令直接從命令列將資料發佈到主題:

'<args>'是您將傳遞到主題的實際數據,採用您在上一節中剛剛發現的結構。


\begin{verbatim}
    ros2 topic pub <topic_name> <msg_type> '<args>'
    ros2 topic pub --once /turtle1/cmd_vel \ 
        geometry_msgs/msg/Twist \ 
        "{linear: {x: 2.0, y: 0.0, z: 0.0},\ 
        angular: {x: 0.0, y: 0.0, z: 1.8}}"
\end{verbatim}

以上我們我們了解如何查尋主題的發布的內容,與如何對主題輸入資料。


\section{參數}
參數是節點的配置值。您可以將參數視為節點設定。
節點可以將參數儲存為整數、浮點數、布林值、字串和列表。
在ROS 2中,每個節點維護自己的參數。有關參數的更多背景信息,請參閱概念文件。

\subsection{param 可以使用的指令}
\begin{enumerate}
\item     delete   :刪除參數
\item  describe :顯示有關聲明參數的描述信息
\item  dump     :以 YAML 檔案格式顯示節點的所有參數
\item  get      :取得參數
\item  list     :輸出可用參數列表
\item  load     :載入節點的參數文件
\item  set      :設定參數
\end{enumerate}


\subsection{list}
可以用list來看不同節的參數
\begin{verbatim}
    ros2 param list
    -------------------------
    /teleop_turtle:
        qos_overrides./parameter_events.publisher.depth
        qos_overrides./parameter_events.publisher.durability
        qos_overrides./parameter_events.publisher.history
        qos_overrides./parameter_events.publisher.reliability
        scale_angular
        scale_linear
    use_sim_time
    /turtlesim:
        background_b
        background_g
        background_r
        qos_overrides./parameter_events.publisher.depth
        qos_overrides./parameter_events.publisher.durability
        qos_overrides./parameter_events.publisher.history
        qos_overrides./parameter_events.publisher.reliability
        use_sim_time
\end{verbatim}

\subsection{get}
若要顯示參數的類型和目前值,請使用下列命令:

<node\_name>與<parameter\_name>可以用list來查詢
\begin{verbatim}
    ros2 param get <node_name> <parameter_name>
    ros2 param get /turtlesim background_g
    ---------------------------------------
    Integer value is: 86
\end{verbatim}
\subsection{set}
我們也可以修改參數的數值
\begin{verbatim}
    ros2 param set <node_name> <parameter_name> <value>
    ros2 param set /turtlesim background_r 150
    ---------------------------------------
    Set parameter successful
\end{verbatim}
\subsection{dump}
您可以使用以下命令查看節點目前的所有參數值:

\begin{verbatim}
    ros2 param dump <node_name>
    ros2 param dump /turtlesim > turtlesim.yaml
    --------------------------------------------
    /turtlesim:
    ros__parameters:
        background_b: 255
        background_g: 86
        background_r: 150
        qos_overrides:
        /parameter_events:
            publisher:
            depth: 1000
            durability: volatile
            history: keep_last
            reliability: reliable
        use_sim_time: false
\end{verbatim}
上面的範例有 > turtlesim.yaml,這樣會輸出檔案如果要看內容可以用cat。

\subsection{load}
您可以使用以下命令將參數從檔案載入到目前正在運行的節點:

\begin{verbatim}
    ros2 param load <node_name> <parameter_file>
    ros2 param load /turtlesim turtlesim.yaml
\end{verbatim}

\subsection{使用時就加入參數}
若要使用已儲存的參數值啟動相同節點,請使用:
\begin{verbatim}
    ros2 run <package_name> <executable_name>\ 
        --ros-args --params-file <file_name>

    ros2 run turtlesim turtlesim_node \ 
        --ros-args --params-file turtlesim.yaml
\end{verbatim}

以上我們學會如何查詢參數與如何修改參數。


\section{動作}
動作是 ROS 2 中的通訊類型之一,適用於長時間運行的任務。

它們由三部分組成:
\begin{enumerate}
    \item 目標
    \item 回饋
    \item 結果
\end{enumerate}

行動建立在 \textbf{主題}和\textbf{服務}的基礎上。它們的功能與服務類似,只是可以取消操作。它們還提供穩定的回饋,而不是返回單一回應的服務。

\subsection{使用動作}
操作過程如下
\begin{enumerate}
    \item 啟動兩個turtlesim節點
    \item 使用turtle\_teleop\_key來控制烏龜
    \item 觀察teleop\_turtle的資訊
\end{enumerate}


\subsubsection{開啟模擬}
\begin{verbatim}
    ros2 run turtlesim turtlesim_node
    ros2 run turtlesim turtle_teleop_key
\end{verbatim}
\subsubsection{控制並觀察}
根據提示去抄做烏龜的轉向。
\begin{verbatim}
    Use arrow keys to move the turtle.
Use G|B|V|C|D|E|R|T keys to rotate to absolute orientations. 
'F' to cancel a rotation.
\end{verbatim}
如果完成就會回傳動作完成
\begin{verbatim}
    [INFO] [turtlesim]: Rotation goal completed successfully
\end{verbatim}
當動作被中止也會有終止訊息
\begin{verbatim}
    [INFO] [turtlesim]: Rotation goal canceled
\end{verbatim}
過程中修改母標的提示
\begin{verbatim}
    [WARN] [turtlesim]: Rotation goal received before a 
        previous goal finished. 
        Aborting previous goal
\end{verbatim}

\subsection{list}
我們可以用list來查詢action有哪些
\begin{verbatim}
    ros2 action list
    ros2 action list -t
\end{verbatim}

\subsection{info}
用info來查詢動作的訊息
\begin{verbatim}
    ros2 action info /turtle1/rotate_absolute
    -----------------------------------------
    Action: /turtle1/rotate_absolute
    Action clients: 1
        /teleop_turtle
    Action servers: 1
        /turtlesim
\end{verbatim}

\subsection{interface}
使用interface show來查詢相關資料格式。
\begin{verbatim}
    ros2 interface show turtlesim/action/RotateAbsolute
    ---------------------------------------------------
# The desired heading in radians
    float32 theta
    ---
# The angular displacement in radians to the starting position
    float32 delta
    ---
# The remaining rotation in radians
    float32 remaining
\end{verbatim}

\subsection{send\_goal}
我們可以用send\_goal把目標傳道action。
\begin{verbatim}
    ros2 action send_goal <action_name> <action_type> <values>
    ros2 action send_goal /turtle1/rotate_absolute \
        turtlesim/action/RotateAbsolute "{theta: 1.57}"
\end{verbatim}

\section{rqt console查看日誌}
\input{./part1/sections/rqt\_console.tex}
\section{啟動節點}
使用命令列工具一次啟動多個節點。

在大多數介紹性教學中,您一直在為運行的每個新節點開啟新終端。當您建立越來越複雜的系統並同時運行越來越多的節點時,打開終端機並重新輸入配置詳細資訊變得乏味。

啟動檔案可讓您同時啟動和設定多個包含 ROS 2 節點的可執行檔。

使用該命令運行單一啟動檔案將立即啟動整個系統 - 所有節點及其配置。ros2 launch

\subsection{運行啟動文件}
我們可以用multisim.launch.py來啟動。
\begin{verbatim}
    ros2 launch turtlesim multisim.launch.py
\end{verbatim}

\subsection{用list來查詢}

我們可以用node list 與 topic list來查訊完整的名稱。

之後我們就可以用pub 或其他的方式來給訊息。
\begin{verbatim}
    ros2 topic pub  /turtlesim1/turtle1/cmd_vel \ 
        geometry_msgs/msg/Twist \ 
        "{linear: {x: 2.0, y: 0.0, z: 0.0}, \ 
        angular: {x: 0.0, y: 0.0, z: 1.8}}"

    ros2 topic pub  /turtlesim2/turtle1/cmd_vel \ 
        geometry_msgs/msg/Twist \ 
        "{linear: {x: 2.0, y: 0.0, z: 0.0}, \ 
        angular: {x: 0.0, y: 0.0, z: -1.8}}"
\end{verbatim}


\section{記錄和回放數據}
記錄關於某個主題發布的數據,以便您可以隨時重播和檢查它。

\subsection{啟動節點}
記的用不同的terminal啟動節點
\begin{verbatim}
    ros2 run turtlesim turtlesim_node
    ros2 run turtlesim turtle_teleop_key
\end{verbatim}
我們也建立一個新目錄來儲檔

\subsection{記入的方法}
要注意topic的名稱要正確,可以用topic list來查詢。
\begin{verbatim}
    ros2 bag record <topic_name>
    ros2 bag record /turtle1/cmd_vel
\end{verbatim}

檔案存的地方會是運行指令的地方這要注意

\subsection{info}
一樣我們可以用info來查詢錄的檔案。

\begin{verbatim}
    ros2 bag info <bag_file_name>
\end{verbatim}
\subsection{play}
可以用play來重現錄製時做的動作。
\begin{verbatim}
    ros2 bag play <bag_file_name>
\end{verbatim}


\part{客戶端庫使用}
客戶端函式庫是允許使用者實作 ROS 2 程式碼的 API。使用客戶端程式庫,使用者可以存取 ROS 2 概念,例如節點、主題、服務等。
\section{colcon建立環境}
colcon是ros的建構工具。
\subsection{安裝colcon}
我們安專colcon記的往後對於會影響python的環境都要小心
\begin{verbatim}
sudo apt install python3-colcon-common-extensions
\end{verbatim}

\subsection{建立工作目錄}
ROS工作空間是具有特定結構的目錄。通常有一個src子目錄。此子目錄內是 ROS 套件的源代碼所在的位置。通常目錄一開始是空的。

\begin{verbatim}
    mkdir -p ~/ros2_ws/src
    cd ~/ros2_ws
    ---------------------------
    .
    |- src

    1 directory, 0 files
\end{verbatim}
\subsection{下載範例程式}
從github下載程式到src目錄中。
\begin{verbatim}
    git clone https://github.com/ros2/examples \ 
        src/examples -b humble
\end{verbatim}
\subsection{編譯}
之後就可以編譯。
\begin{verbatim}
    colcon build
    colcon test
\end{verbatim}

\subsection{環境啟動}
編譯完成後就,就有運作環境可以使用。

\begin{verbatim}
    source install/setup.zsh
\end{verbatim}
\subsection{測試}
\begin{verbatim}
    ros2 run examples_rclcpp_minimal_subscriber \ 
        subscriber_member_function

    ros2 run examples_rclcpp_minimal_publisher \ 
        publisher_member_function
\end{verbatim}


\section{工作目錄}
建立一個工作區並學習如何設定用於開發和測試的覆蓋層。

工作空間是包含 ROS 2 套件的目錄。在使用 ROS 2 之前,有必要在您計劃使用的終端機中取得 ROS 2 安裝工作區。這使得 ROS 2 的軟體包可供您在該終端中使用。

\begin{enumerate}
    \item 建立工作的資料夾
    \item 移動到src把程式下載在此
    \item 回到工作區的根目錄
    \item 檢查依賴
    \item 建構程式
\end{enumerate}

\subsection{建構前的準備}
在建構之前記的要先把程式下載好。
\begin{verbatim}
    mkdir -p ~/ros2_ws/src
    cd ~/ros2_ws/src
    git clone https://github.com/ros/ros_tutorials.git -b humble
    cd ../ 
    rosdep install -i --from-path src --rosdistro humble -y
\end{verbatim}
第一次的時後要記得初始化。
\begin{verbatim}
    rosdep install -i --from-path src --rosdistro humble -y

    ERROR: your rosdep installation has not been initialized 
    yet.  Please run:

        sudo rosdep init
        rosdep update
\end{verbatim}
如國沒又問題就會有下面的訊息
\begin{verbatim}
    All required rosdeps installed successfully
\end{verbatim}

\subsection{覆蓋工作區}
覆蓋層中的包將覆蓋底層中的包。還可以有多層底層和覆蓋層,每個連續的覆蓋層都使用其父底層的包。

您可以turtlesim透過編輯turtlesim視窗上的標題列來修改覆蓋層。
請turtle\_frame.cpp在 中找到該檔案~/ros2\_ws/src/ros\_tutorials/turtlesim/src。

turtle\_frame.cpp使用您喜歡的文字編輯器開啟。

\begin{enumerate}
    \item 找到turtle\_frame.cpp
    \item 修改Title
    \item 建構
    \item 執行比較
\end{enumerate}

\begin{verbatim}
    cd ~/ros2_ws/src/ros_tutorials/turtlesim/src/turtlesim/
    vi turtle_frame.cpp 
    # 修改記的存檔
    colcon build
    source install/local_setup.bash
    ros2 run turtlesim turtlesim_node
\end{verbatim}


\section{編寫一個簡單的發布者和訂閱者 (Python)}
\input{./part2/sections/publisher\_and\_subscriber.tex}


\end{document}
