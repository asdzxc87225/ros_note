\subsection{ros指令}
ros 的指令可用下面的功能
\begin{enumerate}
    \item action - 用於處理與動作相關的操作。
    \item bag - 用於處理與rosbag相關的操作。
    \item component - 用於處理與組件相關的操作。
    \item control - 用於處理與控制相關的操作。
    \item daemon - 用於處理與守護程序相關的操作。
    \item doctor - 用於檢查ROS設置和其他潛在問題。
    \item interface - 顯示與ROS接口相關的信息。
    \item launch - 用於運行一個launch文件。
    \item lifecycle - 用於處理與生命周期相關的操作。
    \item multicast - 用於處理與多播相關的操作。
    \item node - 用於處理與節點相關的操作。
    \item param - 用於處理與參數相關的操作。
    \item pkg - 用於處理與包相關的操作。
    \item run - 用於運行一個特定包的可執行文件。
    \item security - 用於處理與安全相關的操作。
    \item service - 用於處理與服務相關的操作。
    \item topic - 用於處理與主題相關的操作。
    \item wtf - 將wtf作為doctor的別名使用。
\end{enumerate}

\subsection{node}
node 有兩個參數可以用
\begin{enumerate}
    \item info  Output information about a node
    \item list  Output a list of available nodes
\end{enumerate}
\subsubsection{info}
ros2 node info [選項] 節點名稱
\begin{enumerate}
    \item -h, --help:顯示幫助信息並退出。
    \item --spin-time SPIN\_TIME:在等待發現時的旋轉時間(僅在不使用已運行的守護程序時適用)。
    \item -s, --use-sim-time:啟用ROS模擬時間。
    \item --no-daemon:不要生成或使用已運行的守護程序。
    \item --include-hidden:顯示隱藏的主題、服務和動作。
\end{enumerate}
\subsubsection{list}
用法:ros2 node list [選項]
\begin{enumerate}
    \item -h, --help:顯示幫助信息並退出。
    \item --spin-time SPIN\_TIME:在等待發現時的旋轉時間(僅在不使用已運行的守護程序時適用)。
    \item -s, --use-sim-time:啟用ROS模擬時間。
    \item --no-daemon:不要生成或使用已運行的守護程序。
    \item -a, --all:顯示所有節點,包括隱藏的節點。
    \item -c, --count-nodes:僅顯示發現的節點數量。
\end{enumerate}

\subsection{topic}
\begin{enumerate}
    \item bw:顯示主題使用的帶寬。
    \item delay:顯示從標頭中的時間戳的主題延遲。
    \item echo:從主題輸出消息。
    \item find:輸出給定類型的可用主題列表。
    \item hz:將平均發布速率打印到屏幕上。
    \item info:打印有關主題的信息。
    \item list:輸出可用主題的列表。
    \item pub:向主題發布消息。
    \item type:打印主題的類型。
\end{enumerate}
\subsubsection{bw}
用法:ros2 topic bw [選項] 主題名稱
\begin{enumerate}
    \item -h, --help:顯示幫助信息並退出。
    \item --window WINDOW, -w WINDOW:計算速率的最大窗口大小,以消息數為單位(默認值:100)。
    \item --spin-time SPIN\_TIME:在等待發現時的旋轉時間(僅在不使用已運行的守護程序時適用)。
    \item -s, --use-sim-time:啟用ROS模擬時間。
\end{enumerate}
\subsubsection{delay}
用法:ros2 topic delay [選項] 主題名稱

選項:

\begin{enumerate}
    \item -h, --help:顯示幫助信息並退出。
    \item --window WINDOW, -w WINDOW:計算速率的窗口大小,以消息數為單位(默認值:10000)。
    \item --spin-time SPIN\_TIME:在等待發現時的旋轉時間(僅在不使用已運行的守護程序時適用)。
    \item -s, --use-sim-time:啟用ROS模擬時間。
\end{enumerate}

\subsubsection{echo}
用法:ros2 topic echo [選項] 主題名稱 [消息類型]

\begin{enumerate}
    \item -h, --help:顯示幫助信息並退出。
    \item --spin-time SPIN\_TIME:在等待發現時的旋轉時間(僅在不使用已運行的守護程序時適用)。
    \item -s, --use-sim-time:啟用ROS模擬時間。
    \item --no-daemon:不要生成或使用已運行的守護程序。
    \item --qos-profile {preset\_profile}:訂閱時使用的質量服務預設配置。
    \item --qos-depth N:訂閱時的隊列大小設置。
    \item --qos-history {history\_setting}:訂閱時的樣本歷史記錄設置。
    \item --qos-reliability {reliability\_setting}:訂閱時的可靠性設置。
    \item --qos-durability {durability\_setting}:訂閱時的耐久性設置。
    \item --csv:將所有遞歸字段用逗號分隔輸出(例如,用於繪圖)。
    \item --field FIELD:回顯消息的選定字段。使用'.'來選擇子字段。
    \item --full-length, -f:對於陣列、字節和長度> '--truncate-length'的字符串,輸出所有元素。
    \item --truncate-length TRUNCATE\_LENGTH, -l TRUNCATE\_LENGTH:將陣列、字節和字符串截斷為指定長度。
    \item --no-arr:不要打印消息的陣列字段。
    \item --no-str:不要打印消息的字符串字段。
    \item --flow-style:以塊樣式打印集合(與csv格式不可用)。
    \item --lost-messages:已棄用,不起作用。
    \item --no-lost-messages:不要報告消息丟失。
    \item --raw:回顯原始二進制表示。
    \item --filter FILTER\_EXPR:用於過濾要打印的消息的Python表達式。
    \item --once:僅打印收到的第一條消息,然後退出。
\end{enumerate}
\subsubsection{find}
用法:ros2 topic find [選項] 主題類型


\begin{enumerate}
    \item -h, --help:顯示幫助信息並退出。
    \item --spin-time SPIN\_TIME:在等待發現時的旋轉時間(僅在不使用已運行的守護程序時適用)。
    \item -s, --use-sim-time:啟用ROS模擬時間。
    \item --no-daemon:不要生成或使用已運行的守護程序。
    \item -c, --count-topics:僅顯示發現的主題數量。
    \item --include-hidden-topics:將隱藏的主題也考慮在內。
\end{enumerate}
\subsubsection{hz}
用法:ros2 topic hz [選項] 主題名稱
\begin{enumerate}
    \item -h, --help:顯示幫助信息並退出。
    \item --window WINDOW, -w WINDOW:計算速率的窗口大小,以消息數為單位(默認值:10000)。
    \item --filter EXPR:僅測量與指定Python表達式匹配的消息。
    \item --wall-time:使用實際時間計算速率,在模擬期間未發佈時可以很有用。
    \item --spin-time SPIN\_TIME:在等待發現時的旋轉時間(僅在不使用已運行的守護程序時適用)。
    \item -s, --use-sim-time:啟用ROS模擬時間。
\end{enumerate}
\subsubsection{info}
用法:ros2 topic info [選項] 主題名稱

\begin{enumerate}
    \item -h, --help:顯示幫助信息並退出。
    \item --spin-time SPIN\_TIME:在等待發現時的旋轉時間(僅在不使用已運行的守護程序時適用)。
    \item -s, --use-sim-time:啟用ROS模擬時間。
    \item --no-daemon:不要生成或使用已運行的守護程序。
    \item --verbose, -v:打印詳細信息,如節點名稱、節點命名空間、主題類型、GUID以及訂閱者和發布者的QoS配置文件。
\end{enumerate}

\subsection{service}
用法:ros2 service [選項] <命令>

選項:
\begin{enumerate}
    \item -h, --help:顯示幫助信息並退出。
    \item --include-hidden-services:將隱藏的服務也考慮在內。
\end{enumerate}
命令:
\begin{enumerate}
    \item call:調用一個服務。
    \item find:輸出給定類型的可用服務列表。
    \item list:輸出可用服務的列表。
    \item type:輸出服務的類型。
\end{enumerate}
\subsection{pkg}
\subsection{rum}
