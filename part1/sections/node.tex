ROS 中的每個節點都應負責單一的模組化目的,例如控制車輪馬達或發布來自雷射測距儀的感測器資料。每個節點都可以透過主題、服務、操作或參數從其他節點發送和接收資料。

完整的機器人系統由許多協同工作的節點組成。在 ROS 2 中,單一可執行檔(C++ 程式、Python 程式等)可以包含一個或多個節點。

\subsection{ros2 直行方法}
run 的使用方式如下
\begin{verbatim}
    ros2 run <package_name> <executable_name>
    -------------------------------------------
    usage: ros2 run [-h] [--prefix PREFIX]
    package_name executable_name ...

    Run a package specific executable

    positional arguments:
    package_name     Name of the ROS package
    executable_name  Name of the executable
    argv             Pass arbitrary arguments to the executable

    options:
    -h, --help       show this help message and exit
    --prefix PREFIX  Prefix command, which should go before the
    executable. Command must be wrapped in quotes
    if it contains spaces (e.g. --prefix 'gdb -ex
    run --args').
\end{verbatim}
操作流程如下
\begin{enumerate}
    \item 開啟第一個terminal
    \item 開啟turtlesim turtlesim\_node
    \item 開啟第二個terminal
    \item 檢查節點裝況
\end{enumerate}
開啟模擬視窗
\begin{verbatim}
    ros2 run <package_name> <executable_name>
    ros2 run turtlesim turtlesim_node
\end{verbatim}

檢查節點
\begin{verbatim}
    ros2 node list
\end{verbatim}
開啟控制器
\begin{verbatim}
    ros2 run turtlesim turtle_teleop_key
\end{verbatim}
\subsection{重新映射}
重新映射可讓您將預設節點屬性(例如節點名稱、主題名稱、服務名稱等)重新指派給自訂值。
\begin{verbatim}
ros2 run turtlesim turtlesim_node --ros-args --remap \
__node:=my_turtle
\end{verbatim}
這時後會看到我們映射的節點
\begin{verbatim}
    /my_turtle
    /turtlesim
    /teleop_turtle
\end{verbatim}

\subsection{節點訊息}
    可以用info 來查詢節點的資料,方法如下。
\begin{verbatim}
ros2 node info <node_name>
ros2 node info /my_turtle
\end{verbatim}
可以看到如下的連線資料
\begin{enumerate}
    \item  Subscribers:訂閱者
    \item  Publishers:發布者
    \item  Service Servers:服務伺服器
    \item  Service Clients:服務客戶端
    \item  Action Servers:動作伺服器
    \item  Action Clients:動作客戶端
\end{enumerate}
\subsubsection{訂閱者(Subscribers):}

訂閱者是ROS中的一個元件,它可以接收特定主題(Topic)的訊息。
主題是ROS中一種訊息傳遞機制,允許節點(ROS中的程式)透過發佈和訂閱訊息進行通訊。
\subsubsection{發佈者(Publishers):}

發佈者是ROS中的一個元件,它能夠向特定主題發佈訊息。
發佈者節點生成訊息並將其發送到相應的主題,使其他訂閱者節點能夠接收這些訊息。
\subsubsection{服務伺服器(Service Servers):}

服務伺服器提供了一種不同的通訊機制,稱為服務(Service),允許節點提供某種特定的功能或服務。
當一個節點請求服務時,服務伺服器執行相應的任務,並返回結果。
\subsubsection{服務客戶端(Service Clients):}

服務客戶端是一個節點,它能夠向服務伺服器發送請求,並等待伺服器返回結果。
這種通訊模式通常用於需要特定服務或功能的節點之間的交互。
\subsubsection{動作伺服器(Action Servers):}

動作伺服器是ROS中處理長時間執行任務的一種機制。它允許異步執行,並提供反饋。
與服務不同,動作可以週期性地提供反饋,而不僅僅是單一的請求和回應。
\subsubsection{動作客戶端(Action Clients):}

動作客戶端是一個節點,它可以向動作伺服器發送請求,並接收反饋和結果。
這通常用於需要處理較長時間執行的任務,而服務模型則適用於簡短的請求和回應。
