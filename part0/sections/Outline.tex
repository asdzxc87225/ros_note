
我們的專題目前訂成ros2的自走車,最基本的功能就是要作到可以在
地圖上指定一個點他就會自行前往。

要達到這個功能,就要用到\href{https://navigation.ros.org/}{nav2}的功能,我原先參考的資料有下面。
\begin{itemize}
    \item \href{https://emanual.robotis.com/docs/en/platform/turtlebot3/quick-start/}{turtlebot3} 
    \item \href{https://github.com/dblanding/diy-ROS-robot/tree/main}{diy ros}
\end{itemize}

我原先是打算用turtlebot3的套件但是,價格真的不是很低而且他們為了
後續的升級基本上用到的零件都超出需要的性能,所以我嘗試找有沒有土
炮的版本,就被我找到
\href {https://github.com/dblanding}{dblanding}
大大的板本,我算了一下架個大約會在1w上下,但是有些零件實驗室應該
會有。

所以我們就先以完成dblanding大大的版本為一個小目標之後看時間,如
果我們做太慢,就這樣就好。

\subsection{任務}

根據dblanding的github上的資料,他把整體分成下面幾個部份。
\begin{enumerate}
    \item    wheels:輪子控制相關程序
    \item    odom\_pub:里程計數據發布程序
    \item    robot\_nav:機器人導航程序和配置
    \item    robot\_params:機器人參數配置
    \item    bringup:啟動配置和程序管理
\end{enumerate}

單然這要有ros的概念,很不幸的他是用ros1做的而且在2025ros1就不再
會有更新,所以我們會用ros2的版本。

\subsection{先備知識}
要有以下的概念會比較好。
\begin{itemize}
    \item linux 的指令操作(ssh)
    \item ros2 的通訊概念
    \item python
\end{itemize}
