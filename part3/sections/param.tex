
\begin{lstlisting}[language=Python, caption=publish\_params.py]
    """
Publish robot parameters for various reasons:
    They might be needed by more than one package.
    It might be nice to be able to modify them at runtime.
    It is an obvious place to find them.

See doc at http://wiki.ros.org/rospy/Overview/Parameter%20Server
"""
import rospy

# Robot parameter values
ROBOT_TICKS_PER_REV = 686.4  # 24:1 worm, 26:10 spur, 11 pole encoder magnet
ROBOT_WHEEL_CIRCUMFERENCE = 0.213  # meters
ROBOT_TICKS_PER_METER = int(ROBOT_TICKS_PER_REV / ROBOT_WHEEL_CIRCUMFERENCE)
ROBOT_TRACK_WIDTH = .187  # Wheel Separation Distance (meters)
ROBOT_MIN_PWM_VAL = 80  # Minimum PWM value motors will turn reliably
ROBOT_MAX_PWM_VAL = 255  # Maximum allowable PWM value
ROBOT_MIN_X_VEL = 0.1  # Minimum x velocity robot can manage (m/s)
ROBOT_MAX_X_VEL = 0.25  # Maximum x velocity robot can manage (m/s)
ROBOT_MIN_Z_VEL = 0.8  # Minimum theta-z velocity robot can manage (rad/s)
ROBOT_MAX_Z_VEL = 3.0  # Maximum theta-z velocity robot can manage (rad/s)
ROBOT_MTR_KP = 0.5  # Proportional coeff
ROBOT_MTR_KD = 0.2  # Derivative coeff
ROBOT_MTR_MAX_PID_TRIM = 30  # Max allowable value for PID trim term

# end points of segments of piecewise linear curve in descending order
# where curve relates tick rate (tr) to motor speed (s)
ROBOT_TRS_CURVE = ((892, 240),
                   (578, 140),
                   (360, 100),
                   (142, 80))

param_dict = {
    'ROBOT_TICKS_PER_REV': ROBOT_TICKS_PER_REV,
    'ROBOT_WHEEL_CIRCUMFERENCE': ROBOT_WHEEL_CIRCUMFERENCE,
    'ROBOT_TICKS_PER_METER': ROBOT_TICKS_PER_METER,
    'ROBOT_TRACK_WIDTH': ROBOT_TRACK_WIDTH,
    'ROBOT_MIN_PWM_VAL': ROBOT_MIN_PWM_VAL,
    'ROBOT_MAX_PWM_VAL': ROBOT_MAX_PWM_VAL,
    'ROBOT_MIN_X_VEL': ROBOT_MIN_X_VEL,
    'ROBOT_MAX_X_VEL': ROBOT_MAX_X_VEL,
    'ROBOT_MIN_Z_VEL': ROBOT_MIN_Z_VEL,
    'ROBOT_MAX_Z_VEL': ROBOT_MAX_Z_VEL,
    'ROBOT_TRS_CURVE': ROBOT_TRS_CURVE,
    'ROBOT_MTR_KP': ROBOT_MTR_KP,
    'ROBOT_MTR_KD': ROBOT_MTR_KD,
    'ROBOT_MTR_MAX_PID_TRIM': ROBOT_MTR_MAX_PID_TRIM
    }
    
for key, val in param_dict.items():
    rospy.set_param(key, val)

\end{lstlisting}

\subsection{註解內容}
將機器人參數發布出去有幾個原因:
\begin{enumerate}
    \item 它們可能被多個軟體包需要。
    \item 在運行時修改它們是一件好事。
    \item 它是一個明顯的地方可以找到它們。
\end{enumerate}

\subsection{幾本內容}
作者事先定義各個參數的數值,之後在用字典形式存起來
之後在用items()來將key與value成對的存成元組,之後就
可以用rospy的set\_param()來將數值發布出去。

\clearpage
\subsection{參數表}
其中上面的參數含意如下表
\ref{table:example2}。
\begin{table}[ht]
\centering
\caption{參數表}
\begin{tabular}{|l|l|}
\hline
參數名稱 & 定義 \\
\hline
ROBOT\_TICKS\_PER\_REV & 每轉的編碼器脈衝數(24:1蠕蟲、26:10齒輪、11極磁鐵) \\
\hline
ROBOT\_WHEEL\_CIRCUMFERENCE & 輪子周長(米) \\
\hline
ROBOT\_TICKS\_PER\_METER & 每米的編碼器脈衝數(整數) \\
\hline
ROBOT\_TRACK\_WIDTH & 輪子間的距離(米) \\
\hline
ROBOT\_MIN\_PWM\_VAL & 馬達可靠轉動的最小PWM值 \\
\hline
ROBOT\_MAX\_PWM\_VAL & 允許的最大PWM值 \\
\hline
ROBOT\_MIN\_X\_VEL & 機器人可管理的最小x速度(米/秒) \\
\hline
ROBOT\_MAX\_X\_VEL & 機器人可管理的最大x速度(米/秒) \\
\hline
ROBOT\_MIN\_Z\_VEL & 機器人可管理的最小theta-z速度(弧度/秒) \\
\hline
ROBOT\_MAX\_Z\_VEL & 機器人可管理的最大theta-z速度(弧度/秒) \\
\hline
ROBOT\_MTR\_KP & 馬達控制的比例係數 \\
\hline
ROBOT\_MTR\_KD & 馬達控制的微分係數 \\
\hline
ROBOT\_MTR\_MAX\_PID\_TRIM & PID調整項的最大允許值 \\
\hline
ROBOT\_TRS\_CURVE & 編碼器每秒產生的脈衝數與馬達速度的曲線描述\\
\hline
\end{tabular}
\label{table:example2}
\end{table}

