ros2 有yolov8的套件可以用
\href{https://github.com/mgonzs13/yolov8_ros/tree/main}{yolov8\_ros}。
\subsection{安裝方法}
流程如下,就照指令做就好了。
\begin{tcolorbox}
\begin{verbatim}
 cd ~/ros2_ws/src
 git clone https://github.com/mgonzs13/yolov8_ros.git
 pip3 install -r yolov8_ros/requirements.txt
 cd ~/ros2_ws
 rosdep install --from-paths src --ignore-src -r -y
 colcon build
\end{verbatim}
\end{tcolorbox}
\subsection{基本操作}
\begin{tcolorbox}
\begin{verbatim}
$ ros2 launch yolov8_bringup yolov8.launch.py
\end{verbatim}
\end{tcolorbox}
\subsection{主題與參數}
\begin{itemize}
    \item Topics
    \begin{itemize}
        \item /yolo/detections: Objects detected by YOLO using the RGB images. Each object contains a bounding boxes and a class name. It may also include a mak or a list of keypoints.
        \item /yolo/tracking: Objects detected and tracked from YOLO results. Each object is assigned a tracking ID.
        \item /yolo/debug\_image: Debug images showing the detected and tracked objects. They can be visualized with rviz2.
    \end{itemize}
    \item Parameters
    \begin{itemize}
        \item model: YOLOv8 model (default: yolov8m.pt)
        \item tracker: Tracker file (default: bytetrack.yaml)
        \item device: GPU/CUDA (default: cuda:0)
        \item enable: Wether to start YOLOv8 enabled (default: True)
        \item threshold: Detection threshold (default: 0.5)
        \item input\_image\_topic: Camera topic of RGB images (default: /camera/rgb/image\_raw)
        \item  image\_reliability: Reliability for the image topic: 0=system default, 1=Reliable, 2=Best Effort (default: 2)
    \end{itemize}
\end{itemize}
\subsection{注意事項}
使用時他會去訂閱/camera/rgb/image\_raw的topic所以要把想要做辨識的圖發布到這裡。

\subsection{程式分析}
\subsubsection{yolov8\_node.py}

這段程式碼是一個 ROS 2 節點,用於在圖像中偵測和追蹤物體,並且發布偵測到的物體資訊。
節點使用 YOLOv8 模型進行物體偵測,並透過 ROS 2 訊息將偵測到的物體發布出去。

該節點繼承自 LifecycleNode 類別,實現了節點的生命週期方法
(on\_configure、on\_activate、\\ on\_deactivate、on\_cleanup),
並實現了 enable\_cb 方法來處理啟用/禁用節點的服務請求。
節點還實現了一些輔助方法來解析 YOLOv8 的偵測結果,並將結果轉換為 ROS 2 訊息。

在 main 函數中,節點初始化後手動觸發了配置和啟用操作,
並通過 rclpy.spin(node) 來啟動節點的訊息迴圈,等待訊息的到來和處理。\href{https://github.com/mgonzs13/yolov8_ros/blob/main/yolov8_ros/yolov8_ros/yolov8_node.py}{src}


\begin{itemize}
    \item \_\_init\_\_(self, **kwargs) 
        None: 初始化函式,用於初始化節點的各種參數和屬性。
    \item on\_configure(self, state: LifecycleState) \\
        TransitionCallbackReturn: 配置函式,用於配置節點在啟動前的相關設置。
    \item enable\_cb(self, request, response): 啟用/禁用節點的服務回調函式。
    \item on\_activate(self, state: LifecycleState) \\
        TransitionCallbackReturn: 啟動函式,用於啟動節點並開始執行相應的任務。
    \item on\_deactivate(self, state: LifecycleState) \\
        TransitionCallbackReturn: 停用函式,用於停用節點並進行清理工作。
    \item on\_cleanup(self, state: LifecycleState) \\
        TransitionCallbackReturn: 清理函式,用於在節點結束運行時進行清理工作。
    \item parse\_hypothesis(self, results: Results) \\
        List[Dict]: 解析假設的函式,用於解析物體偵測的結果,返回一個假設列表。
    \item parse\_boxes(self, results: Results)\\
         List[BoundingBox2D]: 解析框框的函式,用於解析物體偵測的結果,返回一個框框列表。
    \item parse\_masks(self, results: Results) \\
        List[Mask]: 解析遮罩的函式,用於解析物體偵測的結果,返回一個遮罩列表。
    \item parse\_keypoints(self, results: Results) \\
        List[KeyPoint2DArray]: 解析關鍵點的函式,用於解析物體偵測的結果,返回一個關鍵點列表。
    \item image\_cb(self, msg: Image) \\
        None: 圖像回調函式,用於接收來自訂閱的圖像訊息並進行物體偵測處理
\end{itemize}

