這部份會去監聽輪組控制時發布的三個主題。
\begin{enumerate}
    \item /right\_ticks 主題,用於接收右輪的編碼器數據。
    \item /left\_ticks 主題,用於接收左輪的編碼器數據。
    \item /act\_vel 主題,用於接收機器人的實際速度信息。
\end{enumerate}
\subsection{用到的套件}
這部份用到的套件如下表\ref{tab:modules_and_usage}
\begin{table}[ht]
\centering
\caption{模組/類別及其用途}
\label{tab:modules_and_usage}
\begin{tabular}{|l|p{10cm}|}
\hline
\textbf{模組/類別} & \textbf{用途} \\
\hline
rospy & 創建 ROS 節點、訂閱和發佈主題等 \\
tf2\_ros & 處理轉換關係 \\
Odometry & 定義機器人的里程計信息 \\
Point & 定義機器人的位置信息 \\
Pose & 定義機器人的姿態信息 \\
Quaternion & 定義機器人的旋轉信息(用四元數表示) \\
Twist & 定義機器人的速度信息 \\
Vector3 & 定義三維向量 \\
TransformStamped & 定義轉換關係的消息類型 \\
Int32 & 定義整數的消息類型 \\
Float32 & 定義浮點數的消息類型 \\
asin & 計算反正弦值 \\
sin & 計算正弦值 \\
cos & 計算餘弦值 \\
pi & π 值 \\
\hline
\end{tabular}
\end{table}

\subsection{變數初始化}
這段程式碼是用來初始化 ROS 节点的,並且從 ROS 參數伺服器中獲取了兩個重要的參數值:每米的編碼器脈衝數和車輪間的距離。接著定義了一些變量,包括左輪和右輪的編碼器脈衝數、機器人的線速度(米/秒)和角速度(弧度/秒)。這些變量將用於計算機器人的里程計數據。

\subsection{回調函數}
這段程式碼定義了兩個回調函數 left\_tick\_callback 和
right\_tick\_callback,分別用於接收來自左輪和右輪編碼器的訊息。
這兩個函數將分別更新全局變量 left\_ticks 和 right\_ticks 的
值,以便後續計算里程。接著,定義了兩個函數 
left\_tick\_listener 和 right\_tick\_listener,用於訂閱 ROS
訊息主題 /left\_ticks 和 /right\_ticks,並將它們與對應的
回調函數關聯起來,以便接收編碼器訊息。

回調函數 act\_vel\_callback,用於從 Twist 訊息中提取 linear.x
和 angular.z 的值。這兩個值將被存儲在全局變量 vx 和 vth 中,
以便後續計算里程。接著,定義了一個函數 act\_vel\_listener,用
於訂閱 ROS 訊息主題 /act\_vel,並將其與 act\_vel\_callback 回調
函數關聯起來,以接收實際速度訊息。

\subsection{主迴圈}

這段程式碼包含了主要的里程計計算和發佈邏輯。在一個循環中,
它首先計算了兩個輪子的行進距離,然後根據兩輪之間的差異計算
機器人的轉向角度。接著,它使用這些數據更新機器人的位置和姿勢。
最後,它發佈了里程計訊息和 TF2 的 Transform。

需要注意的是,在計算轉向角度時,程式碼使用了 asin 函數。
但是,這可能會產生 ValueError,因為當 
cycle\_diff / TRACK\_WIDTH 的絕對值大於1時,asin 的參數
必須在 -1 和 1 之間。當發生這種情況時,程式碼將 
cycle\_angle 設置為0,並打印出一條消息,並繼續運行。

此外,程式碼還設置了里程計和速度訊息的協方差矩陣。這些值
通常是根據機器人的實際性能和環境來調整和優化的。最後,它使
用 publish 方法發佈里程計訊息,並使用 r.sleep() 使循環按
照設置的頻率運行。

程式碼是一個循環,持續運行直到ROS被關閉。在每個循環中,它計
算了自上一次循環以來的時間差,並根據左右輪的編碼器讀數計算
了機器人的行進距離和角度變化。然後,它使用這些數據來更新機
器人的位姿(位置和方向),並將該信息發布到ROS中的/odom主題
上,以便其他節點可以使用。最後,它更新了上一次循環的時間,
並通過r.sleep()等待下一個循環。這個循環使機器人能夠持續跟
踪自身的運動狀態,並在ROS中提供里程計信息。

