rqt\_console是一個 GUI 工具,用於在 ROS 2 中內省日誌訊息。通常,日誌訊息會顯示在您的終端機中。使用rqt\_console,您可以隨著時間的推移收集這些訊息,以更有條理的方式仔細查看它們,過濾它們,保存它們,甚至重新加載保存的文件以在不同時間進行反思。


\begin{enumerate}
    \item 開啟rqt
    \item 開啟節點
    \item 控制烏龜
    \item 查看或是存檔
\end{enumerate}

\subsection{開啟rqt}
我們操作之前可以先開rqt,一邊做可以一邊檢查。
\begin{verbatim}
    ros2 run rqt_console rqt_console
\end{verbatim}
\subsection{開啟節點並動作}
開啟節點後用另外一個terminal 來控制。
\begin{verbatim}
    ros2 topic pub -r 1 /turtle1/cmd_vel \ 
        geometry_msgs/msg/Twist \ 
        "{linear: {x: 2.0, y: 0.0, z: 0.0}, \ 
        angular: {x: 0.0,y: 0.0,z: 0.0}}"
\end{verbatim}

\subsection{訊息級別}
ROS 2 的記錄器等級依嚴重性排序:
\begin{enumerate}
    \item    Fatal:訊息表明系統將終止以嘗試保護自身免受損害。
    \item    Error:訊息表明存在重大問題會阻止其正常運作。
    \item  Warn:訊息顯示意外的活動或不理想的結果,可能代表更深層的問題。
    \item  Info:訊息指示事件和狀態更新。    
    \item Debug:息詳細說明了系統執行的整個逐步過程。
\end{enumerate}

\subsection{設定預設記錄器級別}
您可以在首次使用重新映射運行節點時設定預設記錄器等級/turtlesim。在終端機中輸入以下命令:


\begin{verbatim}
    ros2 run turtlesim turtlesim_node --ros-args --log-level WARN
\end{verbatim}
