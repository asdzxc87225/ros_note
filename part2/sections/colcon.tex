colcon是ros的建構工具。
\subsection{安裝colcon}
我們安專colcon記的往後對於會影響python的環境都要小心
\begin{verbatim}
sudo apt install python3-colcon-common-extensions
\end{verbatim}

\subsection{建立工作目錄}
ROS工作空間是具有特定結構的目錄。通常有一個src子目錄。此子目錄內是 ROS 套件的源代碼所在的位置。通常目錄一開始是空的。

\begin{verbatim}
    mkdir -p ~/ros2_ws/src
    cd ~/ros2_ws
    ---------------------------
    .
    |- src

    1 directory, 0 files
\end{verbatim}
\subsection{下載範例程式}
從github下載程式到src目錄中。
\begin{verbatim}
    git clone https://github.com/ros2/examples \ 
        src/examples -b humble
\end{verbatim}
\subsection{編譯}
之後就可以編譯。
\begin{verbatim}
    colcon build
    colcon test
\end{verbatim}

\subsection{環境啟動}
編譯完成後就,就有運作環境可以使用。

\begin{verbatim}
    source install/setup.zsh
\end{verbatim}
\subsection{測試}
\begin{verbatim}
    ros2 run examples_rclcpp_minimal_subscriber \ 
        subscriber_member_function

    ros2 run examples_rclcpp_minimal_publisher \ 
        publisher_member_function
\end{verbatim}

