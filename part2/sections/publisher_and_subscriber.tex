建立節點,這些節點透過主題以字串訊息的形式相互傳遞訊息。這裡使用的例子是一個簡單的「說話者」和「傾聽者」系統;一個節點發布數據,另一個節點訂閱該主題,以便它可以接收該數據。

\subsection{建構包}
打開一個新終端並取得 ROS 2 安裝的源代碼,以便ros2命令可以運行。

\begin{verbatim}
ros2 pkg create --build-type ament_python --license Apache-2.0 py_pubsub
\end{verbatim}

\subsection{編寫發布者節點}
建構完成後前往py\_pubsub/py\_pubsub寫程式,這裡用其他大大提供的原始碼。

\begin{verbatim}
    wget https://raw.githubusercontent.com/ros2/\ 
        examples/humble/rclpy/topics/minimal_publisher/\ 
        examples_rclpy_minimal_publisher/\ 
        publisher_member_function.py

    wget https://raw.githubusercontent.com/ros2/examples/\ 
        humble/rclpy/topics/minimal_subscriber/\ 
        examples_rclpy_minimal_subscriber/\ 
        subscriber_member_function.py
\end{verbatim}

\subsection{新增依賴}
修改package.xml,在裡面加入相依的套件。

\begin{verbatim}
    <description>Examples of minimal publisher/subscriber using rclpy</description>
    <maintainer email="you@email.com">Your Name</maintainer>
    <license>Apache License 2.0</license>
    <exec_depend>rclpy</exec_depend>
    <exec_depend>std_msgs</exec_depend>

    <exec_depend>rclpy</exec_depend>
    <exec_depend>std_msgs</exec_depend>
\end{verbatim}

\subsection{新增入口點}
入口點(entry\_points) talker 與 listener 是用使用時調用的名稱,等號右邊的是執行的程式檔名與要執行的funtion。

\begin{verbatim}
    entry_points={
        'console_scripts': [
                'talker = py_pubsub.publisher_member_function:main',
                'listener = py_pubsub.subscriber_member_function:main',
        ],
    },
\end{verbatim}

\section{建構並運行}
完成上面的設定之後就可以執行看看
\begin{verbatim}
    colcon build --packages-select py_pubsub
    source install/setup.bash
    ros2 run py_pubsub talker
    ros2 run py_pubsub listener
\end{verbatim}

\subsection{程式碼}
\begin{verbatim}
import rclpy
from rclpy.node import Node

from std_msgs.msg import String


class MinimalPublisher(Node):

    def __init__(self):
        super().__init__('minimal_publisher')
        self.publisher_ = self.create_publisher(String, 'topic', 10)
        timer_period = 0.5  # seconds
        self.timer = self.create_timer(timer_period, self.timer_callback)
        self.i = 0

    def timer_callback(self):
        msg = String()
        msg.data = 'Hello World: %d' % self.i
        self.publisher_.publish(msg)
        self.get_logger().info('Publishing: "%s"' % msg.data)
        self.i += 1


def main(args=None):
    rclpy.init(args=args)

    minimal_publisher = MinimalPublisher()

    rclpy.spin(minimal_publisher)

    # Destroy the node explicitly
    # (optional - otherwise it will be done automatically
    # when the garbage collector destroys the node object)
    minimal_publisher.destroy_node()
    rclpy.shutdown()


if __name__ == '__main__':
    main()
\end{verbatim}
\begin{verbatim}
import rclpy
from rclpy.node import Node

from std_msgs.msg import String


class MinimalSubscriber(Node):

    def __init__(self):
        super().__init__('minimal_subscriber')
        self.subscription = self.create_subscription(
            String,
            'topic',
            self.listener_callback,
            10)
        self.subscription  # prevent unused variable warning

    def listener_callback(self, msg):
        self.get_logger().info('I heard: "%s"' % msg.data)


def main(args=None):
    rclpy.init(args=args)

    minimal_subscriber = MinimalSubscriber()

    rclpy.spin(minimal_subscriber)

    # Destroy the node explicitly
    # (optional - otherwise it will be done automatically
    # when the garbage collector destroys the node object)
    minimal_subscriber.destroy_node()
    rclpy.shutdown()


if __name__ == '__main__':
    main()
\end{verbatim}
