建立一個工作區並學習如何設定用於開發和測試的覆蓋層。

工作空間是包含 ROS 2 套件的目錄。在使用 ROS 2 之前,有必要在您計劃使用的終端機中取得 ROS 2 安裝工作區。這使得 ROS 2 的軟體包可供您在該終端中使用。

\begin{enumerate}
    \item 建立工作的資料夾
    \item 移動到src把程式下載在此
    \item 回到工作區的根目錄
    \item 檢查依賴
    \item 建構程式
\end{enumerate}

\subsection{建構前的準備}
在建構之前記的要先把程式下載好。
\begin{verbatim}
    mkdir -p ~/ros2_ws/src
    cd ~/ros2_ws/src
    git clone https://github.com/ros/ros_tutorials.git -b humble
    cd ../ 
    rosdep install -i --from-path src --rosdistro humble -y
\end{verbatim}
第一次的時後要記得初始化。
\begin{verbatim}
    rosdep install -i --from-path src --rosdistro humble -y

    ERROR: your rosdep installation has not been initialized 
    yet.  Please run:

        sudo rosdep init
        rosdep update
\end{verbatim}
如國沒又問題就會有下面的訊息
\begin{verbatim}
    All required rosdeps installed successfully
\end{verbatim}

\subsection{覆蓋工作區}
覆蓋層中的包將覆蓋底層中的包。還可以有多層底層和覆蓋層,每個連續的覆蓋層都使用其父底層的包。

您可以turtlesim透過編輯turtlesim視窗上的標題列來修改覆蓋層。
請turtle\_frame.cpp在 中找到該檔案~/ros2\_ws/src/ros\_tutorials/turtlesim/src。

turtle\_frame.cpp使用您喜歡的文字編輯器開啟。

\begin{enumerate}
    \item 找到turtle\_frame.cpp
    \item 修改Title
    \item 建構
    \item 執行比較
\end{enumerate}

\begin{verbatim}
    cd ~/ros2_ws/src/ros_tutorials/turtlesim/src/turtlesim/
    vi turtle_frame.cpp 
    # 修改記的存檔
    colcon build
    source install/local_setup.bash
    ros2 run turtlesim turtlesim_node
\end{verbatim}

