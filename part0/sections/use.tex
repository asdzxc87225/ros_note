使用方式有幾個東西要啟動
\begin{enumerate}
    \item 機器啟動
        \begin{itemize}
            \item 光學雷達
            \item 馬達區動
            \item urdf 與 tf樹
        \end{itemize}
    \item 導航算法
        \begin{itemize}
            \item slam
            \item nav套件
        \end{itemize}
\end{enumerate}
\subsection{機器啟動}
依序下這幾個指令來請動機器人。
\begin{tcolorbox}
\begin{verbatim}
    ros2 launch sllidar_ros2 sllidar_a1_launch.py
    ros2 launch project display.launch.py
    ros2 launch project wheel_launch.py
\end{verbatim}
\end{tcolorbox}
\subsubsection{注意事項}

起動時要注意ttyUSB的設備,這跟linux的設備管理有關西,先接入的設備是USB0再來是
USB1,所以要注意設備連接的順序。

這些程式要在ros2設備端來啟動。

\subsection{導航算法}
啟動完機器設備後就可以開始啟動算法。
\begin{tcolorbox}
\begin{verbatim}
    ros2 launch nav2_bringup navigation_launch.py
    ros2 launch slam_toolbox online_async_launch.py
\end{verbatim}
\end{tcolorbox}
\subsubsection{注意事項}
要注意啟動算法之錢必須確定tf樹的設定,這個會影響計算的結果且要注意是否有正確指向。
