\subsection{401到407}
輪組的材料檢查
\begin{itemize}
    \item 確認接線方法
    \item 設計個模組間的接線
\end{itemize}
\subsection{408到414}
完成了輪子編碼器設定,馬達的轉速測量
\begin{itemize}
    \item 完成輪組控制電路
    \item 測量控制訊號
    \item 測量霍爾編碼器訊號
\end{itemize}
\subsection{415到421}
本週期中考基本上沒有進度,主裝材料改成用材料行的盒子。
\begin{itemize}
    \item 完成pid調整
    \item 完成輪組控制鐵三角(目標、控制、里程)
    \item 3d列硬大失敗
    \item 修改外觀材料
\end{itemize}
\subsection{422到428}
以完成基本的框架,現在可以進行基本的控制slam與導航。
\begin{itemize}
    \item 車子的前後左右的控制
    \item 光達的slam製圖
    \item nav2導航
\end{itemize}

\subsubsection{問題}
還是jetson的兼容問題,主要是官方沒有提供22.04的版本所以我們從官方提供的最新18.04進行
升級,但是還是有某些軟體不會跟著所以再把我筆點的程式與ros2的套件轉移到jetson nano來
運作時就時常會有軟體相依性的問題。

再來可能是官方已經擺爛了所以20.04或是往後感覺他的驅動都可能會有問題,這次arduino nano
在根jetson nano 做uart的通訊時,也出現亂碼很明縣就是有大量的雜訊干擾雖然反覆的交叉
比對之後最後有把通訊問題排除。

usb wifi 非常不穩定,我在用ssh登入jetson時很卡並且運算上感覺不是很好。


