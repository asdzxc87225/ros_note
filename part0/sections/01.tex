
\subsection{2-25 到 3-02}
本週拿到jetson nano了。
\begin{itemize}
    \item jetson nano
\end{itemize}
\subsubsection{完成任務}
本週住要將jection的作業環境搭建完成,只是nvidia提供的版本
是,ubuntu18.04的版本這不是我們要的版本,所以提升到22.04
但是應為套件的關西更新根升級的時間會花很久。
\begin{itemize}
    \item 安裝 jetpack4.6板
    \item 將內部作業系統升級22.04
    \item usb 與ssh通訊測試
    \item 完成3D列硬的參數測試
\end{itemize}
\subsubsection{遇到的困難}
jetson nano已經停產,所以nvidia沒有為nano做軟體的更新服務
所以jetpack可以用的版本停在4.6板,再上去的版本沒有直接
支援,但這部份只是光方沒寫支援實際我還沒測過,本次的作法
是用jetpack4.6內的ubuntu18.04直接升級到22.04,目前還沒遇到
兼容性問題。
\begin{itemize}
    \item nvidia 軟體支援差
\end{itemize}

\subsection{3-3 到 3-9}
這周零件陸陸續續的收到了,現在手邊的東西有。

\begin{itemize}
    \item 光達
    \item 馬達驅動器
    \item jetson nano
        n nano以今停產,
    \item 不斷電系統
    \item imu 
    \item 航空電池
    \item usb to ttl
    \item 輪子
\end{itemize}

\subsubsection{完成任務}
這周主要收到光達,以測試光達為主。

\subsubsection{遇到的困難}
剛開始的時候雖然官方有提供驅動,所以輕鬆的可以看到/scan
的訊息,但是我在rviz2上只能在laser上看到訊號,在其他座標
係就看不到了,這主要是我的tf觀念有問題要。

\begin{itemize}
    \item 注意tf id
    \item 靜態廣播
    \item urdf設定
\end{itemize}
\subsection{3-10 到 3-14}
\begin{itemize}
    \item if 樹設定
    \item urdf模型設定
\end{itemize}
\subsection{3-15 到 3-21}
\begin{itemize}
    \item 光學雷達里程計 
    \item slam 製圖
\end{itemize}
\subsection{3-22 到 3-28}
\begin{itemize}
    \item 確認馬達相關參數
    \item ros2馬達控制節點
    \item ros2馬達serial節點
    \item ros2馬達里程節點
\end{itemize}
