控制馬達我們要注意的幾個部份
\begin{enumerate}
    \item 控制訊息的格式
    \item arduino 與ros 之間的通訊
    \item PID控制器
    \item 編碼器計算
    \item 里程計計算
\end{enumerate}

\subsection{控制訊息格式}
下面有兩個本次在輪組控制時用到的格式,一個ros內建的用於表示機器在空間中的狀態,
如線性方向的變化(x,y,z)與旋轉變化的(x,y,z),另外一個是我自訂意的變數格式用於表示兩個
輪子的轉動速度(rps)
\begin{itemize}
    \item geometry\_msgs/msg/Twist
    \item diff\_robot\_description/msg/Wheel
\end{itemize}
\begin{tcolorbox}
\begin{verbatim}
ros2 interface show diff_robot_description/msg/Wheel        
--------------------------------------------------------
float64 left_wheel_speed   # 左輪轉速
float64 right_wheel_speed  # 右輪轉速
\end{verbatim}
\end{tcolorbox}
\begin{tcolorbox}
\begin{verbatim}
ros2 interface show geometry_msgs/msg/Twist
-------------------------------------------------------------------------------
# This expresses velocity in free space broken into its linear and angular parts.
Vector3  linear
	float64 x
	float64 y
	float64 z
Vector3  angular
	float64 x
	float64 y
	float64 z
\end{verbatim}
\end{tcolorbox}


\subsection{arduino與ros通訊}
我們這次是用uart通訊來做arduino與ros之間的數據交換。
ros可以用python 的pyserial套件來做urat通訊,arduino用serial就可以了。
\begin{itemize}
    \item ros給arduino目標轉速
    \item arduino給ros測量轉速
\end{itemize}

\subsection{PID}
PID馬達控制透過比例、積分、微分控制器調節馬達輸出,
使實際速度接近目標速度。比例控制速度誤差、
積分控制積累誤差、微分控制速度變化率,綜合調整可達到
精準控制效果。可以參考
\href{https://geekeeceebee.com/index.html}{PID馬達控制}

本次使用
\href{https://github.com/br3ttb/Arduino-PID-Library/tree/master}{arduino pid}
的套件

\begin{itemize}
    \item Setpoint:目標轉速
    \item Input:測量轉速
    \item Output:pwm輸出量
\end{itemize}
